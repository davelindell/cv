% arara: latexmk: {options: ['-c']}
% arara: pdflatex
% arara: biber
% arara: biber
% arara: pdflatex: {shell: yes}
% arara: pdflatex: {shell: yes}

\documentclass[10pt,letter,sans]{moderncv}
\DeclareOldFontCommand{\rm}{\normalfont\rmfamily}{\mathrm}
\DeclareOldFontCommand{\sf}{\normalfont\sffamily}{\mathsf}
\DeclareOldFontCommand{\tt}{\normalfont\ttfamily}{\mathtt}
\DeclareOldFontCommand{\bf}{\normalfont\bfseries}{\mathbf}
\DeclareOldFontCommand{\it}{\normalfont\itshape}{\mathit}
\DeclareOldFontCommand{\sl}{\normalfont\slshape}{\@nomath\sl}
\DeclareOldFontCommand{\sc}{\normalfont\scshape}{\@nomath\sc}
\DeclareRobustCommand*\cal{\@fontswitch\relax\mathcal}
\DeclareRobustCommand*\mit{\@fontswitch\relax\mathnormal}

\moderncvstyle{classic}
\moderncvcolor{blue}
\definecolor{color2}{rgb}{0.25,0.25,0.25}
%\renewcommand{\familydefault}{\sfdefault}

\usepackage[top=1in, bottom=1in, left=1.0in, right=1.0in]{geometry}
\setlength{\hintscolumnwidth}{3.5cm}
%\setlength{\makecvtitlenamewidth}{10cm}
%\frenchspacing

\usepackage[backend=biber,style=ieee,sorting=ydnt,defernumbers=true,dashed=false]{biblatex}

\AtDataInput{%
  \ifcsundef{bbx@processedentries:\therefsection}
    {\csgdef{bbx@processedentries:\therefsection}{}}
    {}%
  \xifinlistcs{\thefield{entrykey}}{bbx@processedentries:\therefsection}{}{%
    \listcsxadd{bbx@processedentries:\therefsection}{\thefield{entrykey}}%
    \csnumgdef{bbx@entrycount:\therefsection:\thefield{keywords}}{%
      \csuse{bbx@entrycount:\therefsection:\thefield{keywords}}+1}}}

% Print the labelnumber as the total number of entries in the
% current refsection, minus the actual labelnumber, plus one
\DeclareFieldFormat{labelnumber}{\mkbibdesc{#1}}    
\newrobustcmd*{\mkbibdesc}[1]{%
  \number\numexpr\csuse{bbx@entrycount:\therefsection:\thefield{keywords}}+1-#1\relax}

% Make the author's name bold in publications.
\newbibmacro{name:format}{%
  \ifthenelse{\equal{\namepartfamily}{Lindell}}%
    {\textbf{\ifblank{\namepartgiveni}{}{\namepartgiveni\space}\namepartfamily}}%
    {\ifblank{\namepartgiveni}{}{\namepartgiveni\space}\namepartfamily}%
  \ifthenelse{\value{listcount}=1 \AND \value{liststop}=2}%
    {\space and}%
    {\ifthenelse{\value{listcount}<\value{liststop}}%
      {\addcomma}%
      {}%
    }%
}
\DeclareNameFormat{author}{
   \usebibmacro{name:format}%
}

\defbibenvironment{bibliography}
 {\list
 {\printtext[labelnumberwidth]{%
    \printfield{labelprefix}%
    \printfield{labelnumber}}}
    {\setlength{\labelwidth}{100pt}%{\labelnumberwidth}%
 \setlength{\leftmargin}{108pt}%{\labelwidth}%
 \setlength{\labelsep}{\biblabelsep}%
 %\addtolength{\leftmargin}{\labelsep}%
 \setlength{\itemsep}{\bibitemsep}%
 \setlength{\parsep}{\bibparsep}}%
}
 {\endlist}
 {\item}


\usepackage{lastpage}
%\cfoot{\addressfont\textcolor{gray}{Page \thepage\ of \pageref{LastPage}}}

\AtEndPreamble{\hypersetup{pdftitle={CV -- David B. Lindell}}}
\name{David B.}{Lindell\vspace{2mm}}
\title{Curriculum Vitae}
\address{1 Vista Montana, Apt 4344}{San Jose, CA 95134}{}
\mobile{+1~507~514~2491}
\email{lindell@cs.toronto.edu}
\homepage{davidlindell.com}
%\social[linkedin]{davelindell}
%\social[twitter]{davelindell}
%\social[github]{davelindell}
%\extrainfo{\small Generated on \today}
%\photo[64pt][0.4pt]{picture}
% \quote{Generated on \today}

\begin{document}
\makecvtitle

\section{Education \& Experience}
\cventry{7/2022--}{Asst. Professor}{Dept. of Computer Science}{University of Toronto}{Toronto, ON}{}
\cventry{1/2021--4/2022}{Postdoctoral Scholar}{Electrical Engineering}{Stanford University}{Stanford, CA}{}
\cventry{9/2016--1/2021}{Ph.D. Student}{Electrical Engineering}{Stanford University}{Stanford, CA}{Advisor: Gordon Wetzstein}
\cventry{9/2015--4/2016}{M.Sc. Student}{Electrical Engineering}{Brigham Young University}{Provo, UT}{Advisor: David G. Long} 
\cventry{9/2009--4/2015}{B.Sc. Student}{Electrical Engineering}{Brigham Young University}{Provo, UT}{Advisors: David G. Long, Aaron Hawkins}

\section{Awards}
\cvitem{2021}{ACM SIGGRAPH Outstanding Doctoral Dissertation Honorable Mention} 
\cvitem{2020}{ACM SIGGRAPH Thesis Fast Forward Honorable Mention} 
\cvitem{2020}{CVPR Outstanding Reviewer} 
\cvitem{2016--2020}{Stanford Graduate Research Fellowship} 
\cvitem{2015}{BYU Office of Research \& Creative Activities Grant} 
\cvitem{2014}{Tau Beta Pi Scholarship} 
\cvitem{2012--2015}{BYU Heritage Scholarship} 

\section{Service}
\cvitem{\bf{Finance Co-Chair}}{Int. Conference on Computational Photography (ICCP) 2022}
\cvitem{\bf{Program Chair}}{CVPR Workshop on Computational Cameras and Displays (CCD) 2021}
\cvitem{}{CVPR Workshop on Computational Cameras and Displays (CCD) 2020}
\cvitem{\bf{Program Committee}}{Int. Conference on Computational Photography (ICCP) 2019--2021}
\cvitem{\bf{Paper Reviewer}}{Nature, Nature Communications, Nature Photonics, Science Advances, SIGGRAPH, TPAMI, CVPR, ECCV, ICCV, TCI, ICCP, Optics Express, NeurIPS} 
\cvitem{\bf{Member}}{ACM, IEEE}

\section{Teaching}
\cvitem{Co-Instructor}{Computational Imaging, EE367/CS448i (Stanford 2022)}
\cvitem{Teaching Assistant}{Computational Imaging, EE367/CS448i (Stanford 2020)}
\cvitem{Instructor/Organizer}{Computational Time-Resolved Imaging, Single-Photon Sensing and Non-Line-of-Sight Imaging (ACM SIGGRAPH 2020)}
% \cvitem{Presenter}{Imaging and Microscopy Reading Group (Stanford 2016--2020)}

\section{Internships}
\cventry{6/2018--11/2018}{Intern}{Intelligent Systems Lab}{Intel Corporation}{Santa Clara, CA}{Advisor: Vladlen Koltun}
\cventry{6/2016--7/2016}{Intern}{}{Rincon Research Corporation}{Tucson, AZ}{}

%\renewcommand*{\bibfont}{\small}
\newpage
\section{Journal Publications}
\begin{refsection}[publications/journal.bib]
  \newrefcontext[labelprefix=J]
  \nocite{*}
  \printbibliography[heading=none]
\end{refsection}
\section{Conference Publications}
\begin{refsection}[publications/conference.bib]
  \newrefcontext[labelprefix=C]
  \nocite{*}
  \printbibliography[heading=none]
\end{refsection}
%\section{Non-Refereed Publications}
%\begin{refsection}[publications/preprints.bib]
%  \newrefcontext[labelprefix=P]
%  \nocite{*}
%  \printbibliography[heading=none]
%\end{refsection}

\section{Theses}
    \cvitem{2021}{Computational Imaging with Single-Photon Detectors. Ph.D. Thesis.}
    \cvitem{2016}{Arctic Sea Ice Classification and Soil Moisture Estimation Using Microwave Sensors. Master's Thesis.}

\section{Public Demonstrations}
    \cventry{2018}{\bf Real-time non-line-of-sight imaging}{M. O'Toole, D.B. Lindell, G. Wetzstein}{2018}{ACM SIGGRAPH Emerging Technologies}{}
    \cventry{2018}{\bf Real-time non-line-of-sight imaging}{M. O'Toole, D.B. Lindell, G. Wetzstein}{2018}{IEEE Conference on Computer Vision and Pattern Recognition (CVPR)}{}

\section{Invited Talks}
    \cvitem{2022}{Implicit Neural Representation Networks for Fitting Signals, Derivatives, and Integrals, Silicon Valley ACM SIGGRAPH Chapter, Virtual.} % 1/20/2022
    \cvitem{2021}{Implicit Neural Representation Networks for Fitting Signals, Derivatives, and Integrals, Samsung AI Centre, Virtual.} % 10/15/2021
    \cvitem{2021}{Implicit Neural Representation Networks for Fitting Signals, Derivatives, and Integrals, University of Erlangen-Nuremberg, Virtual.} % 9/14/2021
    \cvitem{2021}{Physics-Based Visual Computing for Efficient 3D Vision and Sensing, University of Michigan, Virtual.} % 3/29/2021
    \cvitem{2021}{Physics-Based Visual Computing for Efficient 3D Vision and Sensing, MIT RLE, Virtual.} % 3/18/2021
    \cvitem{2021}{Physics-Based Visual Computing for Efficient 3D Vision and Sensing, University of Chicago, Virtual.} % 3/15/2021
    \cvitem{2021}{Physics-Based Visual Computing for Efficient 3D Vision and Sensing, University of Toronto, Virtual.} % 3/9/2021
    \cvitem{2021}{Physics-Based Visual Computing for Efficient 3D Vision and Sensing, Texas A\&M, Virtual.} % 3/5/2021
    \cvitem{2021}{AutoInt: Automatic Integration for Fast Neural Volume Rendering, Google, Virtual.} % 1/19/2021
    \cvitem{2020}{Implicit Neural Representation Networks for Fitting Signals, Derivatives, and Integrals, Graphics and Mixed Environment Seminar (GAMES), Virtual.} % 12/17/2020
    \cvitem{2020}{A Camera to See Around Corners, Playground/Akasha Imaging, Palo Alto, CA.} % 2/5/2020
    \cvitem{2019}{A Camera to See Around Corners, TEDxBeaconStreet, Boston, MA.} %11/23/2019
    \cvitem{2019}{Computational Imaging with Single-Photon Detectors, Boston University Center for Information \& Systems Engineering (CISE), Boston, MA.} %11/22/2019
    \cvitem{2019}{Efficient Confocal Non-Line-of-Sight Imaging, MIT RLE, Cambridge, MA.} %11/22/2019
    \cvitem{2019}{Efficient Confocal Non-Line-of-Sight Imaging, MIT Media Lab, Cambridge, MA.} %11/21/2019
    \cvitem{2019}{Computational Imaging with Single-Photon Detectors, Berkeley Center for Computational Imaging, Berkeley, CA.} %9/4/2019
    \cvitem{2019}{Computational Single-Photon Imaging, Silicon Valley ACM SIGGRAPH Chapter, San Jose, CA.} %5/30/2019
    \cvitem{2019}{Computational Imaging with Single-Photon Detectors, Stanford Center for Image Systems Engineering (SCIEN), Stanford, CA.} %5/8/2019
    \cvitem{2019}{Computational Single-Photon Imaging, Carnegie Mellon University Graphics Lab, Pittsburgh, PA.} %1/23/2019

\section{Mentorship}
    \cventry{\bf Ph.D.}{Axel Levy}{Stanford}{Fall 2021}{}{}
    \cventry{}{Dave Van Veen}{Stanford}{Fall 2021}{}{}
    \cventry{}{William Meng}{Stanford}{Summer 2021}{}{}
    \cventry{}{Qingqing Zhao}{Stanford}{Fall 2020}{}{}
    \cventry{}{Manu Gopakumar}{Stanford}{Fall 2020}{}{}
    \cventry{}{Thomas Teisberg}{Stanford}{Fall 2019}{}{}
    \cventry{}{Alex Bergman}{Stanford}{Summer 2019}{}{}
    \cventry{}{Mark Nishimura}{Stanford}{Summer 2019}{}{}
    \cventry{}{Zhanghao Sun}{Stanford}{Winter 2019}{}{}
    \cventry{\bf High School}{Jason Corona}{South San Francisco High School CA}{2019--2020}{}{}
  \vspace{-5mm}

%\vspace{5mm}
%\section{Graduate Coursework}
%\cvlistitem{
%    Machine Learning (CS-229),
%    A. Ng
%    \hfill
%    F2018
%}
%\cvlistitem{
%    Convex Optimization (EE-364A),
%    S. Boyd
%    \hfill
%    Sp2017
%}
%\cvlistitem{
%    Convolutional Neural Networks for Visual Recognition (CS-231N),
%    F. Li
%    \hfill
%    Sp2017
%}
%\cvlistitem{
%    Computational Imaging and Display (EE-367),
%    G. Wetzstein
%    \hfill
%    W2017
%}
%\cvlistitem{
%    Information Theory (EE 376),
%    D. Tse
%    \hfill
%    W2017
%}
%\cvlistitem{
%    The Fourier Transform and its Applications (EE-261),
%    B. Osgood
%    \hfill
%    F2016
%}
%\cvlistitem{
%    Linear Dynamical Systems (EE-263),
%    R.N. Mahalati
%    \hfill
%    F2016
%}
%\cvlistitem{
%    Detection and Estimation Theory (EE-672),
%    M. Rice
%    \hfill
%    W2016
%}
%\cvlistitem{
%    Continuous Phase Modulation (EE-682R),
%    M. Rice
%    \hfill
%    W2016
%}
%\cvlistitem{
%    Robotic Vision (EE-631),
%    D.J. Lee
%    \hfill
%    W2016
%}
%\cvlistitem{
%    Math of Signals and Systems (EE-671),
%    B. Jeffs
%    \hfill
%    F2015
%}
%\cvlistitem{
%    Stochastic Processes (EE-670),
%    B. Mazzeo
%    \hfill
%    F2015
%}
%\cvlistitem{
%    Medical Imaging \& Image Reconstruction (EE-576),
%    N. Bangerter
%    \hfill
%    F2015
%}
%\cvlistitem{
%    Antennas and Propogation (EE-665),
%    K. Warnick
%    \hfill
%    W2015
%}
%\cvlistitem{
%    Microwave Remote Sensing (EE-568),
%    D. Long
%    \hfill
%    F2014
%}
%  \vspace{-5mm}

\end{document}
